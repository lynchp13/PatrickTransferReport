% $Log: abstract.tex,v $
% Revision 1.1  93/05/14  14:56:25  starflt
% Initial revision
% 
% Revision 1.1  90/05/04  10:41:01  lwvanels
% Initial revision
% 
%
%% The text of your abstract and nothing else (other than comments) goes here.
%% It will be single-spaced and the rest of the text that is supposed to go on
%% the abstract page will be generated by the abstractpage environment.  This
%% file should be \input (not \include 'd) from cover.tex.

% should be ~500 words

%% Background
Grasping moving objects is a challenging problem in the field of robotics. Traditional approaches are computationally expensive, over reliant on vision-based systems and often require an existing sensor infrastructure in the robot's environment. These make existing techniques unsuitable for implementation on mobile robotic platforms. The potential contribution of alternative strategies, in particular of other types of sensing, remains under-explored. 
%% Aims
The aim of this project is to investigate the role which sensing plays in the ability of a robot to grasp an object in motion. It is hypothesised that a sensor-heavy approach, where multiple sensing modalities are combined, can enable a robust grasping strategy, suited to implementation on a mobile robot.
%% Method
In order to test this hypothesis, the topic was broken into several research questions and addressed individually. In order to enable testing, a two-finger pincer gripper was designed and manufactured. To date, tests were performed on this pincer gripper with and without tactile sensing and their performances compared. A similar methodology will be employed in the future to investigate other sensing mechanisms, including vision and remote tactile.
%% Results and Conclusion
Tests have found that tactile feedback provided in real-time enabled the gripper to rapidly adapt to the moving object. This resulted in higher observed grasp robustness, in particular when the object's trajectory was offset from the center of the gripper. These findings indicate that tactile sensing can play an important role in informing the gripper motion when grasping moving objects and motivates further research on how the contribution of tactile sensing can be optimised and what other types of sensing might be utilised in creating a sensor-informed grasping motion for autonomous grasping of moving objects.

% conclusion, comment, contribution
% how have you added to field?



