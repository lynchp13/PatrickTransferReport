% $Log: abstract.tex,v $
% Revision 1.1  93/05/14  14:56:25  starflt
% Initial revision
% 
% Revision 1.1  90/05/04  10:41:01  lwvanels
% Initial revision
% 
%
%% The text of your abstract and nothing else (other than comments) goes here.
%% It will be single-spaced and the rest of the text that is supposed to go on
%% the abstract page will be generated by the abstractpage environment.  This
%% file should be \input (not \include 'd) from cover.tex.

% should be ~500 words
% introduction
A robot's ability to grasp objects vastly increases its potential, allowing it to interact and manipulate its environment. 
% motivation/aims/ research qs
Despite extensive research, autonomously grasping a moving object remains a very challenging problem; this document presents research which aims to tackle this. 
% methods
Three different areas are explored to achieve this, a predictive model is developed to estimate where the object will be at a given point in time, enabling a robotic system to intercept it. Robotic hardware (grippers and manipulators) are examined to determine the factors which might effect its ability to grasp a moving object, specifically the contribution of a compliant skin, hand geometry and actuation. Finally different types of sensing are experimented with and their suitability and potential contribution to such a grasping system is determined, there is particular focus on tactile and image sensing with an examination of the control strategies which are enabled when a system has image and/or tactile sensing. 
% results
It was found that a predictive model can be used to calculate an appropriate intercept point with a dynamic object, however such an approach is more reliable when the model is informed by information about the object and what its expected behaviour is. The physical embodiment of the gripper has a huge effect on the success rate of grasps and design features like compliment skin can massively increase the chances of a successful grasp. Finally tactile sensing can be used to enable a smarter grasping strategy while visual servoing techniques prove effective for achieving the interception point with the grasper.
% conclusion, comment, contribution
% how have you added to field?



