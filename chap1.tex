%% This is an example first chapter.  You should put chapter/appendix that you
%% write into a separate file, and add a line \include{yourfilename} to
%% main.tex, where `yourfilename.tex' is the name of the chapter/appendix file.
%% You can process specific files by typing their names in at the 
%% \files=
%% prompt when you run the file main.tex through LaTeX.

% make sure to keep this 2-5 pages

\chapter{Introduction}
Advances in robotic technology have enabled a continuous increase in what a robot is capable of. A constantly growing range of features is increasing the potential usefulness of robotic technology. Robots have already been widely adapted in industrial applications for automation and are becoming increasingly ubiquitous in a wider range of applications. Service robots are one example of this, it is increasingly common to see robots in entertainment, hospitality and health care applications.There are huge advantages to expanding robotic applications into these areas. To name just two examples, robots can be used to massively increase the independence of older adults or people living with disabilities and robots can carry out repetitive, and tedious jobs without boredom affecting their performance. With this ever expanding range of potential application however, new challenges are presented. 

% This paragraph needs rewritten and much of it rephrased and eloborated on

Robots, even relatively primitive robotic platforms, perform very well in the automation of manufacturing and assembly of products in industry and have done for several decades now. This is mainly due to the environment, which, in the case of manufacturing or assembly, is highly controlled and designed specifically for the robot in question. A robot used in automated assembly does not need to be adaptable, it will never have to work in different lighting conditions, or need to move to pick up something which is outside of it's manipulator's range, and it does not have to consider that something, for example a human, might get in the path of its arm. This is not true of robots which are designed to work in less structured environments. A robot working in a hotel, building lobby or health care facility needs to be much more adaptable, this adaptability requires a much higher level of complexity. There is a demand, even within industrial applications, for a more flexible and adaptable robot.  

One area where this move to more unstructured and dynamic environments is particularly challenging is manipulation and grasping. Grasping is an essential function of a general purpose robot. The ability to grasp and manipulate an object extends the robot's ability beyond sensing and navigating its environment to being able to manipulate and interact with it's environment, this massively increases its potential. In a more controlled environment when a robot grasps an object, very often the problem is vastly simplified by ensuring the object is in a known location and in a favourable orientation. When a robots needs to be able to effectively grasp in less structured environment this is not the case. For example, when the robot is more general purpose the range of objects which it will need to have the ability to grasp is much larger. More than just the size of the object will change, but also the shape and weight too. Furthermore, it is naive to assume that the objects which you will want to grasp will always be static and remain static throughout the interaction. This is where the focus of this research lays, this document outlines ongoing research which specifically examines the autonomous robotic grasping of dynamic objects.

\section{Background}

Robotic manipulation refers to a robots ability to manipulate its object in its environment. It is the robotic equivalent a humans hands and arms. This research focuses specifically on the end effector or gripper of the robot. This refers to the mechanism at the end of a robotic arm.

A vital part of grasping any object is the ability to localise it in space. The robot much have information about the relative pose, positions and velocities, of the object and itself. This can be achieved through the use of sensors, both sensing the external world, i.e. visual cameras and depth cameras and those monitoring the state of the robot, i.e. joint sensing, etc. The data collected, when processed can yield the necessary relative pose of robot and object as well as other information about the environment which may be relevant, i.e. possible collisions, etc. The success of this approach is dependant on everything from the lighting conditions and the shape of the object to the algorithm used.

This is further complicated in this work since we are trying to grasp a moving object. This uses a similar approach to locating a static object however it will do it across consecutive frames and collect data about the movement of the object, for example the velocity vector. One example predictive model is a simple linear projection of the balls position at time tnow+time to intercept, based on its current speed and direction.

new position = old position + speed * time to intercept

This research will also explore how further information about the movement of the object might help to inform the predictive model, i.e. a ball rolling down an inclined plane will require a different predictive model than a ball flying through three dimensional space.  

\subsection{Gripper}
The gripper is an essential part of this system, a well designed gripper could massively decrease the demands on other parts of the system, for example the accuracy of the intercept from the predictive model. A poorly designed gripper on the other hand may be fundamentally incapable of grasping a moving object regardless of the rest of the system. For this reason, a robotic grasper was developed specifically for grasping a dynamic object. This design was then validated experimentally and important design features where isolated and their value assessed. This experiment and the corresponding results are outlined below.

\subsection{Sensing}
Effective sensing is vital for any robotic system, sensors are how robots collect information about the world around it, they allow them to model, manipulate and react to their environment. In this research there is particular focus on two types of sensing, tactile and visual.

Visual sensing in this application is used for the localisation of the object and for the predictive model. Vision processing algorithms are implemented used OpenCV.

Tactile sensing is embedded in the hand and is used to inform the control of the hand enabling a smarter grasping strategy.


% research questions
\section{Research Questions}
\begin{itemize}
\item Can a vision based predictive model operate at a sufficiently high frame rate to allow an effective object interception % This can be answered by looking at prior art

\item Is there an approach to the predictive model which always performs best or do we need further information about the potential behaviour of the object % This is essentially trivial, It its fair to assume that more information about the behaviour will lead to a better predictive model

\item Can a manipulator, gripper system react sufficiently fast to information provided by a predictive model to grasp a moving object % Change this to refer to sensing, what modes of sensing give inform which is useful to inform the grasp

\item What features of a gripper design facilitate the grasping of a moving object %More specific about features

\item Can embedded tactile sensing enable a smarter grasping strategy and offer an increase in successful grasp % Too similar to the point above
\end{itemize}

% aims and hypothesis
\section{Aims and Objectives}
The primary aims of this research include:

\begin{itemize}
\item Present gazebo simulations which comparing the effectiveness of different approaches to predictive modelling of a dynamic object in different situations.
\item Present the results of experimental testing, conducted to examine how different features of a gripper contribute to a grippers ability to grasp a moving object.
\item Present a grasping strategy for the control of robot fingers based on feedback from embedded tactile sensors
\item Present gazebo simulations of effective visual servoing 
\end{itemize}

% aims 

% research contributions

% objectives



% document outline
\section{Document Outline}
Chapter 2 will present a comprehensive outline of the existing research and associated literature relevant to this challenge and give a sense of the current state of the art. Chapter 3 will explore predictive modelling of the dynamic object. This will be done in simulation where, several different approaches can be applied to a range of situations relatively quickly. Chapter 4 will present controlled experiments and corresponding results of robot grippers ability to grasp objects, different features of the gripper design are isolated and their contribution to the grasp assessed. Chapter 5 will look at sensing, both tactile and visual. The range of possibilities of each are explored and their value assessed. Finally Chapter 6 will conclude this transfer report, summing up the finding to date and outlining future work for this research project.



