% make sure to keep this 2-5 pages

%       Introduction
%
%   Opening Arguments
%       Justify Research
%
%   Introduces the Research
%   

\chapter{Introduction}

%% General paragraph introducing robotics as the topic and the trend towards applying them to new domains. Mentions service robots
Robots are becoming increasingly ubiquitous in a wide range of applications. Advances in robotics enable a growing range of features and increase the potential usefulness of the technology. Having been widely adapted in industrial applications for automation, there is increasing demand to apply robots elsewhere. Service robots are an example of this, it is increasingly common to see robots in entertainment, hospitality and health care applications. There are significant advantages associated with deploying robots in these domains. For example, robots can be used to increase the independence of older adults or people living with disabilities. However, with this ever expanding range of potential applications new challenges are presented. 

% What is the problem and why is it challenging, Structured vs unstructured environments

Robots, even relatively primitive platforms, perform well in highly-structured environments. Automation of manufacturing and assembly of products in industry using robots is largely a solved problem. In the case of manufacturing or assembly, the environment in which the robot is deployed is highly controlled and often designed specifically for the robot in question. A robot deployed for the automation of an assembly process does not need to be robust to lighting conditions, need to grasp something which is outside of it's manipulator's range, or consider that something, for example a human, might get in the path of its arm. This is not true of robots which are designed to work in less structured environments. A robot working in a hotel, building lobby or health care facility needs to be more adaptable. This adaptability requires a much higher level of complexity. There is a demand for a more flexible, adaptable robot.  

%% Introducing grasping moving objects

One area where this move to more unstructured and dynamic environments is particularly challenging is manipulation and grasping. Manipulation and grasping are essential functions of a general purpose robot. The ability to grasp and manipulate an object extends the robot's ability beyond sensing and navigating its environment to the ability to manipulate and interact with it's environment. In a controlled environment the problem is typically simplified by ensuring that a known object is in a known location and favourable orientation. When a robots needs to be able to effectively grasp in less structured environments this is not the case. Furthermore, the range of objects which a general purpose service robot would encounter and be expected to grasp is much larger. Finally, it is naive to assume that the objects which you will want to grasp will always be static and remain static throughout the interaction. This document outlines ongoing research which specifically examines the autonomous robotic grasping of dynamic objects.

\section{Background}

Robotic manipulation refers to a robots ability to interact with an object in its environment. It is the robotic equivalent a human using its arms and hands to interact with objects. This research is concerned  specifically with the end effector or gripper of the robot, i.e. mechanism at the end of a robotic manipulator. At a high level, robotic manipulation consists of 3 primary components, sensing, processing/planning and actuation. A wide variety of sensing manipulation and grasping strategies typically are just variations of these 3 core components. Sensing refers to how a robotic arm collects data about its environment, including the target object. Processing or planning refers to how the system analyses that sensory input and decides on an appropriate response. There are a variety of approaches to planning, ranging from reactive strategies to planning strategies, these will be discussed more later. Finally, Actuation is the physical movement of the robotic system in an attempt to achieve the goal.

Grasping is considered by some to be a solved problem \cite{APCObservations}, others have said that it's technology readiness level (TRL) should be considered between a 3 and a 4 \cite{KukayouBot} (TRL is a scale between 1 and 9). This disagreement about the current state of the art in grasping stems from the context in which the grasping is implemented. Grasping in structured environments is largely considered a solved problem. The type of grasping which has be deemed between a 3 and 4 technology readiness level refers to grasping on a professional service robot or a robot in a flexible cognitive manufacturing scenario. In other words robots operating in less structured environments. This is a distinction which is very important in grasping and in this project.


\subsection{Challanges}

Despite the maturity of manipulation and grasping research, there are persistent challenges which researchers have been striving to overcome for decades and are still not satisfactorily solved. The most commonly encountered include:
\begin{itemize}
    \item Unstructured Environments \cite{Delft, Zhao}
    
    Unstructured environments is a problem common to robotics and is mentioned above. It is particularly relevant in grasping where traditionally a complex problem has been simplified though the use of assumptions. In unstructured environments it is not effective to assume there are no obstacles between the robot and target object, or that the object is static. Creating a robotic gripper which is effective in unstructured environments is a very difficult problem.
    \item Occlusion \cite{Delft, Detectionandmapping, Zhao}
    
    Traditionally grasping problems have relied heavily on vision sensing. Vision sensing enables the passive collection of large amounts of accurate data about the environment. Robotics relies heavily on vision for grasping, mostly for the same reasons which humans rely heavily on their vision. Unfortunately, vision sensing is inherently vulnerable to occlusion problems, particularly in grasping applications where the gripper itself tends to occluded any on-board vision sensors. In humans, this shortcoming is overcome by using other senses.
    \item Lighting Conditions \cite{Delft, Detectionandmapping}
    
    Sensitivity to different lighting conditions is a common problem in computer vision. A reliance on vision sensors make it a problem in robotics also. Conditions which are too bright or too dark can often cause problem during manipulation.
    \item Computation
    
    The computational demands of a grasping task range from very simple sensing and motor control to computationally demanding computer vision and path planning algorithms. This is of particular concern on mobile platforms where computation may be a limited resource.
    \item Latency
    
    Latency can pose a significant problem when conducting time critical tasks, such as grasping moving objects. Latency can be introduced at both a software and hardware level. In software there is a non trivial amount of time needed to analysis sensor data and compute an appropriate action. Similarly, when an appropriate action is determined a non-trivial amount of time is required for the robotic system to carry out that action, for example to reach a target position.
    
    
\end{itemize}


% research questions
\section{Research Questions}
This research project aim to address the following research questions.
\begin{itemize}
\item How can the grasping motion of the robotic gripper be optimised to grasp moving objects. 
\item Can tactile sensing enable a reactive grasping strategy and offer an increase in grasping success rate 
\item How can image sensors contribute to informing the grasping motion of a robotic gripper when grasping a moving object
\item Can a neural network be used to couple sensor stimulus to grasping actuators to achieve a more robust grasp
\item Can reinforcement learning be used to train such a neural network
\item Can alternative sensing modalities be used to optimise gripper motion when grasping a moving object
\end{itemize}

% aims and hypothesis
\section{Aims and Objectives}
The primary aims of this research include:

\begin{itemize}
    
\item Present a grasping strategy for the control of robot fingers based on feedback from a) embedded tactile sensors b) remote tactile sensors
\item Train a neural network to take sensor data as an input and control the grasping motion as an output
\item Investigate the role of image sensor placement while grasping a moving object.
\item Propose a visuotactile grasping system capable of robustly grasping moving objects
\end{itemize}

% document outline
\section{Document Outline}
Chapter 2 will present a comprehensive outline of the existing research and associated literature relevant to this challenge and give a sense of the current state of the art. Chapter 3 will present a gripper which was designed and manufactured during this project for the purposes of addressing the aforementioned research questions. Chapter 4 will present a tactile sensor, inspired by literature and created during this project to address research questions related to tactile sensing. Chapter 5 describes controlled experiments which were conducted and their corresponding results. These explore the ability of tactile sensing to enable a reactive gripper motion for grasping moving objects. Chapter 6, supported by several important appendices, will outline the plan for the rest of this research project. Showing granularity such as week by week timeline predictions, outline of experiments, methods and equipment required and plans for publications.



